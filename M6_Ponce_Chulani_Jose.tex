%Archivo latex del proyecto final

%Formato artículo, tamaño de letra 11
\documentclass[11pt,a4paper]{article}


%Todos estos paquetes son los que he usado en el tfg
%Sé que hay muchos, pero por si acaso los pongo
\usepackage[utf8x]{inputenc}
\usepackage{array}
\usepackage{amsmath}
\usepackage{amsbsy}
\usepackage{bm} %Para poner en negrita las letras griegas, tambien sirve el de arriba
\usepackage{upgreek} %Para quitarle la cursiva a las letras griegas, con el comando \up''letra griega'' todo junto
\usepackage{mathrsfs} 
\usepackage{amssymb}
\usepackage{marvosym}
\usepackage{epsfig}
\usepackage{graphics}
\usepackage{graphicx}
\usepackage{amsfonts}
\usepackage{xspace}
\usepackage{color}
\usepackage{booktabs}
\usepackage{xtab}
\usepackage{subfig}
\usepackage{graphicx}

\usepackage[spanish]{babel}

\begin{document}

\title{\'Atomos Ultrafr\'ios}
\author{Jos\'e Ponce Chulani}

\maketitle


\bigskip

\begin{abstract}
  %URL del repositorio
  En este artículo vamos a tratar brevemente un campo de la física cuántica que está en auge: el de los \'atomos y m\'oleculas ultrafr\'ias. Nos centraremos en el modelo de Bose-Hubbard para estudiar algunas de sus propiedades.
\end{abstract}

\bigskip

\textit{Palabras Clave:}
Átomos ultrafríos, modelo Bose-Hubbard, superfluido, aislante Mott, tuneleo.

\\

\begin{abstract}
  %URL del repositorio
  In the current proyect we give a brief review of an investigation field of quantum physics that is very popular nowadays: the ultracold atoms or molecules. Specifically we study some properties of the Bose-Hubbard model. 
\end{abstract}

\bigskip

\textit{Keywords:}
Ultracold atoms, Bose-Hubbard model, suprfluid, Mott insulattor, tunneling.



\section{Introducción}


\section{El sistema}

Como ya hemos mencionado nuestro modelo contiene dos ingredientes básicos: una red óptica y una cierta cantidad de átomos que vamos a cargar en la red. Estos átomos van a ser de carácter bosónico, de manera que tendremos la libertad de almacenar tantos como queramos (o cuanto podamos, mejor dicho) en un mismo estado cuántico. La dinámica de los bosones a lo largo de la red va a depender de dos términos fundamentalmente, la energía cinética de los átomos y de cuán fuerte interaccionen entre ellos, pudiendo dar lugar al famoso efecto túnel cuando la primera energía domine sobre la otra.

\\

La red óptica se logra haciendo interferir dos rayos láser, formando un patrón de pozos potenciales. En este patron se carga el gas de bosones ya enfriado, de manera que un esquema del sistema viene dado por la figura 

Este tipo de redes son muy versátiles: se pueden conseguir muchas geometrías diferentes casi sin defectos, e incluso puede cambiarse su forma y la intensidad de los mínimos de potencial en el curso de un experimento. 




\end{document}
